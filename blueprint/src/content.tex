\chapter{The integers}

We construct the integers starting from the natural numbers $\Z$. Since \texttt{lean} already has a
type called $\Z$, we define a new type called $\MyInt$ that will be another definition of the integers.
$\MyInt$ will be a quotient of a type called $\MyPreint$.

\begin{definition}
    \label{MyPreint}
    \lean{MyPreint}
    \leanok
    Let $\MyPreint$ be $\N \times \N$
\end{definition}

\begin{definition}
    \label{MyPreint.R}
    \lean{MyPreint.R}
    \leanok
    \uses{MyPreint}
We define a relation $R$ on $\MyPreint$ as follows: $(a,b)$ and $(c, d)$ are related if and only if
\[
a + d = b + c
\]
\end{definition}

\begin{lemma}
$R$ is a reflexive relation.
    \label{MyPreint.R_refl}
    \lean{MyPreint.R_refl}
    \leanok
\end{lemma}
\begin{proof}
    \leanok
    \uses{MyPreint.R}
    This follows by commutativity of addition in $\N$.
\end{proof}

\begin{lemma}
$R$ is a symmetric relation.
    \label{MyPreint.R_symm}
    \lean{MyPreint.R_symm}
    \leanok
\end{lemma}
\begin{proof}
    \leanok
    \uses{MyPreint.R}
    This follows by commutativity of addition in $\N$.
\end{proof}

\begin{lemma}
$R$ is a transitive relation.
    \label{MyPreint.R_trans}
    \lean{MyPreint.R_trans}
    \leanok
\end{lemma}
\begin{proof}
    \leanok
    \uses{MyPreint.R}
    Let $x$, $y$ and $z$ in $\MyPreint$ such that $x R y$ and $y R z$. We can write $x = (a,b)$ and similarly
    for $y = (c,d)$ and $z = (e,f)$. By assumption we have $a+d=b+c$ and $c+f=d+e$. Adding these equalities we get
    \[
    a+d+c+f=b+c+d+e
    \]
    Since addition on $\N$ is cancellative we get
    \[
    a+f = b + e
    \]
    as wanted.
\end{proof}

It follows in particular that $R$ is an equivalence relation. From now on, we will write $x \approx y$ for
$x R y$.

\begin{definition}
    \label{MyPreint.neg}
    \lean{MyPreint.neg}
    \leanok
    \uses{MyPreint.R}
We define an operation, called \emph{negation} on $\MyPreint$ as follows: the negation of $x = (a,b)$ is
$(b,a)$:
\[
-x = -(a,b) = (b,a)
\]
\end{definition}

\begin{lemma}
    \label{MyPreint.neg_quotient}
    \lean{MyPreint.neg_quotient}
    \leanok
If $x \approx x'$ then $-x \approx -x'$.
\end{lemma}
\begin{proof}
    \uses{MyPreint.neg}
Let $x = (a,b)$ and $x' = (a',b')$, so by assumption $a + b' = b + a'$. Be definition we have
\[
-x=-(a,b)=(b,a) \mbox{ and } -x'=-(a',b')=(b',a')
\]
We need to show $b+a'=b'+a$, which follows immediately from $a + b' = b + a'$.
\end{proof}

\begin{definition}
    \label{MyPreint.add}
    \lean{MyPreint.add}
    \leanok
    \uses{MyPreint.R}
We define an operation, called \emph{addition} on $\MyPreint$ as follows: the addition of $x = (a,b)$
and $y = (b, c)$ is
\[
x + y = (a,b) + (c,d) = (a+c, b + d)
\]
\end{definition}