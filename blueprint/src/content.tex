\chapter{The integers}

We construct the integers starting from the natural numbers $\Z$. Since \texttt{lean} already has a
type called $\Z$, we define a new type called $\MyInt$ that will be another definition of the integers.

\section{The \emph{preintegers}}

$\MyInt$ will be a quotient of a type called $\MyPreint$.

\begin{definition}
    \label{MyPreint}
    \lean{MyPreint}
    \leanok
    Let $\MyPreint$ be $\N \times \N$
\end{definition}

\begin{definition}
    \label{MyPreint.R}
    \lean{MyPreint.R}
    \leanok
    \uses{MyPreint}
We define a relation $R$ on $\MyPreint$ as follows: $(a,b)$ and $(c, d)$ are related if and only if
\[
a + d = b + c
\]
\end{definition}

\begin{lemma}
$R$ is a reflexive relation.
    \label{MyPreint.R_refl}
    \lean{MyPreint.R_refl}
    \leanok
\end{lemma}
\begin{proof}
    \leanok
    \uses{MyPreint.R}
    This follows by commutativity of addition in $\N$.
\end{proof}

\begin{lemma}
$R$ is a symmetric relation.
    \label{MyPreint.R_symm}
    \lean{MyPreint.R_symm}
    \leanok
\end{lemma}
\begin{proof}
    \leanok
    \uses{MyPreint.R}
    This follows by commutativity of addition in $\N$.
\end{proof}

\begin{lemma}
$R$ is a transitive relation.
    \label{MyPreint.R_trans}
    \lean{MyPreint.R_trans}
    \leanok
\end{lemma}
\begin{proof}
    \leanok
    \uses{MyPreint.R}
    Let $x$, $y$ and $z$ in $\MyPreint$ such that $x R y$ and $y R z$. We can write $x = (a,b)$ and similarly
    for $y = (c,d)$ and $z = (e,f)$. By assumption we have $a+d=b+c$ and $c+f=d+e$. Adding these equalities we get
    \[
    a+d+c+f=b+c+d+e
    \]
    Since addition on $\N$ is cancellative we get
    \[
    a+f = b + e
    \]
    as wanted.
\end{proof}

\begin{lemma}
    \label{MyPreint.R_equiv}
    \lean{MyPreint.R_equiv}
We have that $R$ is an equivalence relation. From now on, we will write $x \approx y$ for
$x R y$.
\end{lemma}
\begin{proof}
    \leanok
    \uses{MyPreint.R_refl, MyPreint.R_symm, MyPreint.R_trans}
Clear from \ref{MyPreint.R_refl}, \ref{MyPreint.R_symm} and \ref{MyPreint.R_trans}.
\end{proof}

\begin{definition}
    \label{MyPreint.neg}
    \lean{MyPreint.neg}
    \leanok
    \uses{MyPreint.R}
We define an operation, called \emph{negation} on $\MyPreint$ as follows: the negation of $x = (a,b)$ is
$(b,a)$:
\[
-x = -(a,b) = (b,a)
\]
\end{definition}

\begin{lemma}
    \label{MyPreint.neg_quotient}
    \lean{MyPreint.neg_quotient}
    \leanok
If $x \approx x'$, then $-x \approx -x'$.
\end{lemma}
\begin{proof}
\uses{MyPreint.neg}
\leanok
Let $x = (a,b)$ and $x' = (a',b')$, so by assumption $a + b' = b + a'$. By definition we have
\[
-x=-(a,b)=(b,a) \mbox{ and } -x'=-(a',b')=(b',a')
\]
We need to show $b+a'=b'+a$, which follows immediately from $a + b' = b + a'$.
\end{proof}

\begin{definition}
    \label{MyPreint.add}
    \lean{MyPreint.add}
    \leanok
    \uses{MyPreint.R}
We define an operation, called \emph{addition} on $\MyPreint$ as follows: the addition of $x = (a,b)$
and $y = (b, c)$ is
\[
x + y = (a,b) + (c,d) = (a+c, b + d)
\]
\end{definition}

\begin{lemma}
    \label{MyPreint.add_quotient}
    \lean{MyPreint.add_quotient}
    \leanok
If $x \approx x'$ and $y \approx y'$, then $x + y \approx x' + y'$.
\end{lemma}
\begin{proof}
\uses{MyPreint.add}
\leanok
Let $x = (a,b)$, $y = (c,d)$, $x' = (a',b')$ and $y' = (c',d')$ such that $x \approx x'$ and $y \approx y'$. by assumption we have
\[
a + b' = b + a' \mbox{ and } c + d' = d + c'
\]
Adding these two equalities we get
\[
a+b'+c+d'=b+a'+d+c'
\]
and hence
\[
a+c+b'+d'=b+d+a'+c'
\]
that is $x + y = (a+c,b+d) \approx (a'+c',b'+d') = x'+y'$.
\end{proof}

\begin{definition}
    \label{MyPreint.mul}
    \lean{MyPreint.mul}
    \leanok
    \uses{MyPreint.R}
We define an operation, called \emph{multiplication} on $\MyPreint$ as follows: the multiplication of $x = (a,b)$ and $y = (b, c)$ is
\[
x * y = (a,b) * (c,d) = (a*c+b*d, a*d+b*c)
\]
\end{definition}

\begin{lemma}
    \label{MyPreint.mul_quotient}
    \lean{MyPreint.mul_quotient}
    \leanok
If $x \approx x'$ and $y \approx y'$, then $x * y \approx x' * y'$.
\end{lemma}
\begin{proof}
\uses{MyPreint.mul}
\leanok
Let $x = (a,b)$, $y = (c,d)$, $x' = (a',b')$ and $y' = (c',d')$ such that $x \approx x'$ and $y \approx y'$. by assumption we have
\[
a + b' = b + a' \mbox{ and } c + d' = d + c'
\]
Multiplying the first equality by $c'$ and by $d'$ we get
\begin{equation} \label{H1}
a * c' + b' * c' = b * c' + a' * c'
\end{equation}
and
\begin{equation} \label{H2}
b * d' + a' * d' = a * d' + b' * d'
\end{equation}
Multiplying the second equality by $a$ and by $b$ we get
\begin{equation} \label{H3}
a * c + a * d' = a * d + a * c'
\end{equation}
and
\begin{equation} \label{H4}
b * d + b * c' = b * c + b * d'
\end{equation}
Adding \eqref{H1} and \eqref{H4} we get
\[
a * c' + b' * c' + b * d + b * c' = b * c' + a' * c' + b * c + b * d'
\]
Adding \eqref{H3} and \eqref{H2} we get
\[
 a * c + a * d' + b * d' + a' * d' = a * d + a * c' + a * d' + b' * d'
\]
Adding the last two equations and cancelling the terms appearing on both sides we finally have
\[
b' * c' + b * d + a * c + a' * d' =  a' * c' + b * c + a * d + b' * d'
\]
that is $x * y \approx  x'*y'$.
\end{proof}

\section{The integers}

\subsection{Definitions}

\begin{definition}
    \label{MyInt}
    \lean{MyInt}
    \leanok
    \uses{MyPreint.R_equiv}
    We define our integers $\MyInt$ as
\[
\MyInt = \MyPreint \, / \approx
\]
We will write $\llbracket (a, b) \rrbracket$ for the class of $(a,b)$.
\end{definition}

\begin{definition}
    \label{MyInt.zero}
    \lean{MyInt.zero}
    \uses{MyPreint.R_equiv}
    \leanok
We define the zero of $\MyInt$, denoted $0$ as the class of $(0,0)$.
\end{definition}

\begin{definition}
    \label{MyInt.one}
    \lean{MyInt.one}
    \uses{MyPreint.R_equiv}
    \leanok
We define the one of $\MyInt$, denoted $1$ as the class of $(1,0)$.
\end{definition}

\subsection{Commutative ring structure}

\begin{definition}
    \label{MyInt.neg}
    \lean{MyInt.neg}
    \uses{MyPreint.R_equiv, MyPreint.neg_quotient}
    \leanok
We define the negation of $x = \llbracket (a, b) \rrbracket$ in $\MyInt$ as
\[
-x = \llbracket -(a, b) \rrbracket
\]
Thanks to Lemma~\ref{MyPreint.neg_quotient} this is well defined.
\end{definition}

\begin{definition}
    \label{MyInt.add}
    \lean{MyInt.add}
    \uses{MyPreint.R_equiv, MyPreint.add_quotient}
    \leanok
We define the addition of $x = \llbracket (a, b) \rrbracket$ and $y = \llbracket (c, d) \rrbracket$ in $\MyInt$ as
\[
x + y = \llbracket (a, b)+(c,d) \rrbracket
\]
Thanks to Lemma~\ref{MyPreint.add_quotient} this is well defined.
\end{definition}

\begin{definition}
    \label{MyInt.mul}
    \lean{MyInt.mul}
    \uses{MyPreint.R_equiv, MyPreint.mul_quotient}
    \leanok
We define the multiplication of $x = \llbracket (a, b) \rrbracket$ and $y = \llbracket (c, d) \rrbracket$ in $\MyInt$ as
\[
x * y = \llbracket (a, b)*(c,d) \rrbracket
\]
Thanks to Lemma~\ref{MyPreint.mul_quotient} this is well defined.
\end{definition}

\begin{lemma}
    \label{MyInt.add_assoc}
    \lean{MyInt.add_assoc}
    \leanok
    Addition on $\MyInt$ is associative.
\end{lemma}
\begin{proof}
    \uses{MyInt.add}
    \leanok
To prove the lemma it is enough to prove that, for all $a$, $b$, $c$, $d$, $e$ and $f$ in $\N$, we have
\[
(\llbracket (a, b) \rrbracket+ \llbracket (c, d) \rrbracket) + \llbracket (e, f) \rrbracket = \llbracket (a, b) \rrbracket + (\llbracket (c, d) \rrbracket + \llbracket (e, f) \rrbracket)
\]
Unravelling the definitions this is
\[
a + c + e + (b + (d + f)) = b + d + f + (a + (c + e))
\]
that is true by associativity and commutativity in $\N$.
\end{proof}

\begin{proposition}
    \label{MyInt.commRing}
    \lean{MyInt.commRing}
    \leanok
    $\MyInt$ with addition and multiplication is a commutative ring.
\end{proposition}
\begin{proof}
    \uses{MyInt.add, MyInt.mul, MyInt.zero, MyInt.one}
    \leanok
    We have to prove various properties, namely:
    \begin{itemize}
        \item addition is associative (already done in \ref{MyInt.add_assoc})
        \item $0$ works as neutral element for addition (on both sides)
        \item existence of an inverse for addition (we prove that $x + (-x) = (-x) + x = 0$)
        \item addition is commutative
        \item left and right distributivity of multiplication with respect to addition
        \item associativity of multiplication
        \item $1$ works as neutral element for multiplication (on both sides)
    \end{itemize}
    All the proofs are essentially identical to the proof of Lemma~\ref{MyInt.add_assoc} above.
\end{proof}

\begin{lemma}
    \label{MyInt.zero_ne_one}
    \lean{MyInt.zero_ne_one}
    \leanok
In $\MyInt$ we have $0 \neq 1$.
\end{lemma}
\begin{proof}
    \leanok
    \uses{MyInt.zero, MyInt.one}
    If $0 = 1$ by definition we would have $\llbracket (0,0) \rrbracket = \llbracket (1,0) \rrbracket$
    so $0+1=0+0$ in $\N$, that is absurd.
\end{proof}

\begin{lemma}
    \label{MyInt.mul_ne_zero}
    \lean{MyInt.mul_ne_zero}
    \leanok
Let $x$ and $y$ in $\MyInt$ such that $x \neq 0$ and $y \neq 0$. Then $x*y \neq 0$.
\end{lemma}
\begin{proof}
    \leanok
    \uses{MyInt.zero, MyInt.mul}
    It is enough to prove that, for all $a$, $b$, $c$ and $d$ in $\N$ such that $a \neq b$ and $c \neq d$ we have $a*c+b*d\neq a*d+b*c$. If this is not the case we must have $a = b$ or $c = d$. (Can you see how to prove this in $\N$? You cannot use subtraction!) and we are done.
\end{proof}

\subsection{\texorpdfstring{The inclusion $i \colon \N \to \MyInt$}{The inclusion}}
\begin{definition}
    \label{MyInt.i}
    \lean{MyInt.i}
    \uses{MyInt}
    \leanok
    We define a map
\begin{gather*}
    i \colon \N \to \MyInt \\
    n \mapsto \llbracket (n,0) \rrbracket
\end{gather*}
\end{definition}

\begin{lemma}
    \label{MyInt.i_zero}
    \lean{MyInt.i_zero}
    \leanok
We have that $i(0) = 0$.
\end{lemma}
\begin{proof}
    \uses{MyInt, MyInt.i}
    \leanok
Clear from the definition.
\end{proof}

\begin{lemma}
    \label{MyInt.i_one}
    \lean{MyInt.i_one}
    \leanok
We have that $i(1) = 1$.
\end{lemma}
\begin{proof}
    \uses{MyInt, MyInt.i}
    \leanok
Clear from the definition.
\end{proof}

\begin{lemma}
    \label{MyInt.i_add}
    \lean{MyInt.i_add}
    \leanok
For all $a$ and $b$ in $\N$ we have that
\[
i(a+b) = i(a) + i(b)
\]
\end{lemma}
\begin{proof}
    \uses{MyInt.add, MyInt.i}
    \leanok
    We have $i(a+b) = \llbracket (a+b, 0) \rrbracket$, $i(a) = \llbracket (a, 0) \rrbracket$ and
    $i(b) = \llbracket (b, 0) \rrbracket$, so we need to prove that
\[
\llbracket (a+b, 0) \rrbracket = \llbracket (a, 0) \rrbracket + \llbracket (b, 0) \rrbracket
\]
that is obvious from the definition.
\end{proof}

\begin{lemma}
    \label{MyInt.i_mul}
    \lean{MyInt.i_mul}
    \leanok
For all $a$ and $b$ in $\N$ we have that
\[
i(a*b) = i(a) * i(b)
\]
\end{lemma}
\begin{proof}
    \uses{MyInt.mul, MyInt.i}
    \leanok
    We have $i(a+b) = \llbracket (a+b, 0) \rrbracket$, $i(a) = \llbracket (a, 0) \rrbracket$ and
    $i(b) = \llbracket (b, 0) \rrbracket$, so we need to prove that
\[
\llbracket (a+b, 0) \rrbracket = \llbracket (a, 0) \rrbracket * \llbracket (b, 0) \rrbracket
\]
By definition of multiplication we have
\[
\llbracket (a, 0) \rrbracket * \llbracket (b, 0) \rrbracket = \llbracket (a*c+0*0, a*0+0*b) \rrbracket
\]
and the lemma follows.
\end{proof}

\begin{lemma}
    \label{MyInt.i_injective}
    \lean{MyInt.i_injective}
    \leanok
    We have that $i$ is injective.
\end{lemma}
\begin{proof}
    \uses{MyInt.i}
    \leanok
    Let $a$ and $b$ such that $i(a)=i(b)$. This means $\llbracket (a,0) \rrbracket = \llbracket (b,0) \rrbracket$ so $a + 0 = 0 + b$ and hence $a = b$.
\end{proof}

\subsection{The order}

\begin{definition}
    \label{MyInt.le}
    \lean{MyInt.le}
    \uses{MyInt.i, MyInt.add}
    \leanok
Let $x$ and $y$ in $\MyInt$. We write $x \leq y$ if there exist a natural number $n$ such that
\[
y = x + i(n)
\]
\end{definition}

\begin{lemma}
    \label{MyInt.le_refl}
    \lean{MyInt.le_refl}
    \leanok
    The relation $\leq$ on $\MyInt$ is reflexive.
\end{lemma}
\begin{proof}
    \uses{MyInt.le}
    We can just take $n = 0$.
\end{proof}

\begin{lemma}
    \label{MyInt.le_trans}
    \lean{MyInt.le_trans}
    \leanok
    The relation $\leq$ on $\MyInt$ is transitive.
\end{lemma}
\begin{proof}
    \uses{MyInt.le}
    Let $x$, $y$ and $z$ such that $x \leq y$ and $y \leq z$. It follows that there exist $p$ and $q$
    such that $y = x + i(p)$ and $z = y + i(q)$. One can now take $p+q$ to show that $x \leq z$.
\end{proof}

\begin{lemma}
    \label{MyInt.le_antisymm}
    \lean{MyInt.le_antisymm}
    \leanok
    The relation $\leq$ on $\MyInt$ is antisymmetric.
\end{lemma}
\begin{proof}
    \uses{MyInt.le, MyInt.i_zero, MyInt.i_injective}
    Let $x$ and $y$ such that $x \leq y$ and $y \leq x$. It follows that there exist $p$ and $q$
    such that $y = x + i(p)$ and $x = y + i(q)$. In particular
    \[
    x = x + i(p) + i(q)
    \]
    Since $\MyInt$ is a ring, we obtain $i(p)+i(q) = 0$. Moreover $i(p+q)=i(p)+i(q)$ and $i(0)=0$, so
    $i(p+q)=i(0)$ and hence, since $i$ is injective, $p+q = 0$. Now, $p$ and $q$ are \emph{natural numbers}, so $p=q=0$ and so $x=y$.
\end{proof}

It follows that $\leq$ is an order relation.

\begin{lemma}
    \label{MyInt.le_total}
    \lean{MyInt.le_total}
    \leanok
    The order $\leq$ on $\MyInt$ is a total order.
\end{lemma}
\begin{proof}
    \uses{MyInt.le, MyInt.i_zero, MyInt.i_injective}
    \leanok
    Let $x$ and $y$ by in $\MyInt$. We can write $x = \llbracket (a,b) \rrbracket$ and
    $y = \llbracket (c,d) \rrbracket$ and we need to prove that
    $\llbracket (a,b) \rrbracket \leq \llbracket (c,d) \rrbracket$ or $\llbracket (c,d) \rrbracket \leq \llbracket (a,b) \rrbracket$. Let's consider two cases (we use that the order on $\N$ is total):
    \begin{itemize}
        \item if $a + d \leq b + c$ let $e$ be such that $b + c = a + d + e$. We prove that
        $\llbracket (a,b) \rrbracket \leq \llbracket (c,d) \rrbracket$ using $e$. We have
        \[
        \llbracket (a,b) \rrbracket+i(e) = \llbracket (a,b) \rrbracket + \llbracket (e,0) \rrbracket
        = \llbracket (a+e,b+0) \rrbracket = \llbracket (a+e,b) \rrbracket
        \]
        We have that $\llbracket (a+e,b) \rrbracket = \llbracket (c,d) \rrbracket$ since
        \[
        a+e+d=b+d
        \]
        by our assumption on $e$ and so $\llbracket (a,b) \rrbracket \leq \llbracket (c,d) \rrbracket$.
        \item the case $b + c \leq a + d$ is completely analogous.
    \end{itemize}
\end{proof}

\begin{lemma}
    \label{MyInt.linearOrder}
    \lean{MyInt.linearOrder}
    \leanok
    We have that $\MyInt$ with $\leq$ is a \emph{linear order}
\end{lemma}
\begin{proof}
    \uses{MyInt.le_refl, MyInt.le_antisymm, MyInt.le_trans, MyInt.le_total}
    \leanok
    Clear from the lemmas above.
\end{proof}

\begin{lemma}
    \label{MyInt.zero_le_one}
    \lean{MyInt.zero_le_one}
    \leanok
    In $\MyInt$ we have that $0 \leq 1$.
\end{lemma}
\begin{proof}
    \uses{MyInt.zero, MyInt.one, MyInt.le, MyInt.add}
    \leanok
    We use $1$ (as natural number). We need to prove that $0 + i(1) = 1$. Unravelling the definitions
    this is obvious.
\end{proof}

\subsection{Interaction between the order and the ring structure}

\begin{lemma}
    \label{MyInt.add_le_add_left}
    \lean{MyInt.add_le_add_left}
    \leanok
    Let $x$, $y$ and $z$ in $\MyInt$ be such that $x \leq y$. Then $z + x ≤ z + y$.
\end{lemma}
\begin{proof}
    \uses{MyInt.le, MyInt.add}
    \leanok
    Let $n$ be such that $y = x + i(n)$. It's immediate that $n$ also work to show that $z + x ≤ z + y$.
\end{proof}

\begin{lemma}
    \label{MyInt.mul_pos}
    \lean{MyInt.mul_pos}
    \leanok
    Let $x$ and $y$ in $\MyInt$ be such that $0 < x$ and $0 < y$. Then $0 < x * y$.
\end{lemma}
\begin{proof}
    \leanok
    \uses{MyInt.mul_ne_zero, MyInt.le, MyInt.mul}
    By Lemma~\ref{MyInt.mul_ne_zero} we already know that $x*y \neq 0$, so it is enough to prove that
    $0 \leq x*y$. Since $0 < x$, we have in particular that $0 \leq x$, and let $n$ be such that $x = 0 + i(n)$. Similarly, let $m$ be such that $y = 0 + i(m)$. We have $0 + i(n) = i(n)$ and $0 + i(m) = i(m)$, so we need to prove that $0 \leq i(n)*i(m)$. We do so using $n*m$: we have
    \[
    0+i(n*m)=i(n*m)=i(n)*i(m)
    \]
    as required.
\end{proof}

\chapter{The rationals}

We can now define the rationals, starting with the integers. Rather than using our copy of the integers it is convenient to use the ``standard'' version $\Z$ already defined in \Lean{} (so all the relevant results are there).

We follow a similar path to the one for $\MyInt$.

\section{The \emph{prerationals}}

$\MyRat$ will be a quotient of a type called $\MyPrerat$.

\begin{definition}
    \label{MyPrerat}
    \lean{MyPrerat}
    \leanok
    Let $\MyPrerat$ be $\Z \times \Z \backslash \{0\}$
\end{definition}

\begin{definition}
    \label{MyPrerat.R}
    \lean{MyPrerat.R}
    \leanok
    \uses{MyPrerat}
We define a relation $R$ on $\MyPrerat$ as follows: $(a,b)$ and $(c, d)$ are related if and only if
\[
a * d = b * c
\]
\end{definition}

\begin{lemma}
$R$ is a reflexive relation.
    \label{MyPrerat.R_refl}
    \lean{MyPrerat.R_refl}
    \leanok
\end{lemma}
\begin{proof}
    \leanok
    \uses{MyPrerat.R}
    Exercice.
\end{proof}

\begin{lemma}
$R$ is a symmetric relation.
    \label{MyPrerat.R_symm}
    \lean{MyPrerat.R_symm}
    \leanok
\end{lemma}
\begin{proof}
    \leanok
    \uses{MyPrerat.R}
    Exercice.
\end{proof}

\begin{lemma}
$R$ is a transitive relation.
    \label{MyPrerat.R_trans}
    \lean{MyPrerat.R_trans}
    \leanok
\end{lemma}
\begin{proof}
    \leanok
    \uses{MyPrerat.R}
    Exercice.
\end{proof}

\begin{lemma}
    \label{MyPrerat.R_equiv}
    \lean{MyPrerat.R_equiv}
We have that $R$ is an equivalence relation. From now on, we will write $x \approx y$ for
$x R y$.
\end{lemma}
\begin{proof}
    \leanok
    \uses{MyPrerat.R_refl, MyPrerat.R_symm, MyPrerat.R_trans}
Clear from \ref{MyPrerat.R_refl}, \ref{MyPrerat.R_symm} and \ref{MyPrerat.R_trans}.
\end{proof}

\begin{definition}
    \label{MyPrerat.neg}
    \lean{MyPrerat.neg}
    \leanok
    \uses{MyPrerat.R}
We define an operation, called \emph{negation} on $\MyPrerat$ as follows: the negation of $x = (a,b)$ is
$(-a,b)$:
\[
-x = -(a,b) = (-a,b)
\]
Note that it automatically well defined (meaning that second component of $(-a,b)$ is different from $0$).
\end{definition}

\begin{lemma}
    \label{MyPrerat.neg_quotient}
    \lean{MyPrerat.neg_quotient}
    \leanok
If $x \approx x'$, then $-x \approx -x'$.
\end{lemma}
\begin{proof}
\uses{MyPrerat.neg}
\leanok
Exercice.
\end{proof}

\begin{definition}
    \label{MyPrerat.add}
    \lean{MyPrerat.add}
    \leanok
    \uses{MyPrerat.R}
We define an operation, called \emph{addition} on $\MyPrerat$ as follows: the addition of $x = (a,b)$
and $y = (b, c)$ is
\[
x + y = (a,b) + (c,d) = (a * d + b * c, b*d)
\]
Do you see why it is well defined?
\end{definition}

\begin{lemma}
    \label{MyPrerat.add_quotient}
    \lean{MyPrerat.add_quotient}
    \leanok
If $x \approx x'$ and $y \approx y'$, then $x + y \approx x' + y'$.
\end{lemma}
\begin{proof}
\uses{MyPrerat.add}
\leanok
Exercice.
\end{proof}

\begin{definition}
    \label{MyPrerat.mul}
    \lean{MyPrerat.mul}
    \leanok
    \uses{MyPrerat.R}
We define an operation, called \emph{multiplication} on $\MyPrerat$ as follows: the multiplication of $x = (a,b)$ and $y = (b, c)$ is
\[
x * y = (a,b) * (c,d) = (a*c, b*d)
\]
\end{definition}

\begin{lemma}
    \label{MyPrerat.mul_quotient}
    \lean{MyPrerat.mul_quotient}
    \leanok
If $x \approx x'$ and $y \approx y'$, then $x * y \approx x' * y'$.
\end{lemma}
\begin{proof}
\uses{MyPrerat.mul}
\leanok
Exercice.
\end{proof}

\begin{definition}
    \label{MyPrerat.inv}
    \lean{MyPrerat.inv}
    \leanok
    \uses{MyPrerat.R}
We define an operation, called \emph{negation} on $\MyPrerat$ as follows: the inverse of $x = (a,b)$ is:
\[
\mbox{if} b \neq 0 \mbox{ then } x^{-1} = (b, a), \mbox{ otherwise } x^{-1} = (0, 1)
\]
Note that $x^{-1}$ is \emph{always} defined!
\end{definition}

\begin{lemma}
    \label{MyPrerat.inv_quotient}
    \lean{MyPrerat.inv_quotient}
    \leanok
If $x \approx x'$, then $x^{-1} \approx x'^{-1}$.
\end{lemma}
\begin{proof}
\uses{MyPrerat.inv}
\leanok
Exercice.
\end{proof}