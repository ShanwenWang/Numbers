\chapter{The integers}

We construct the integers starting from the natural numbers $\Z$. Since \texttt{lean} already has a
type called $\Z$, we define a new type called $\MyInt$ that will be another definition of the integers.
$\MyInt$ will be a quotient of a type called $\MyPreint$.

\begin{definition}
    \label{MyPreint}
    \lean{MyPreint}
    \leanok
    Let $\MyPreint$ be $\N \times \N$
\end{definition}

\begin{definition}
    \label{MyPreint.R}
    \lean{MyPreint.R}
    \leanok
    \uses{MyPreint}
We define a relation $R$ on $\MyPreint$ as follows: $(a,b)$ and $(c, d)$ are related if and only if
\[
a + d = b + c
\]
\end{definition}

\begin{lemma}
$R$ is a reflexive relation.
    \label{MyPreint.R_refl}
    \lean{MyPreint.R_refl}
    \leanok
\end{lemma}
\begin{proof}
    \leanok
    \uses{MyPreint.R}
    This follows by commutativity of addition in $\N$.
\end{proof}

\begin{lemma}
$R$ is a symmetric relation.
    \label{MyPreint.R_symm}
    \lean{MyPreint.R_symm}
    \leanok
\end{lemma}
\begin{proof}
    \leanok
    \uses{MyPreint.R}
    This follows by commutativity of addition in $\N$.
\end{proof}

\begin{lemma}
$R$ is a transitive relation.
    \label{MyPreint.R_trans}
    \lean{MyPreint.R_trans}
    \leanok
\end{lemma}
\begin{proof}
    \leanok
    \uses{MyPreint.R}
    Let $x$, $y$ and $z$ in $\MyPreint$ such that $x R y$ and $y R z$. We can write $x = (a,b)$ and similarly
    for $y = (c,d)$ and $z = (e,f)$. By assumption we have $a+d=b+c$ and $c+f=d+e$. Adding these equalities we get
    \[
    a+d+c+f=b+c+d+e
    \]
    Since addition on $\N$ is cancellative we get
    \[
    a+f = b + e
    \]
    as wanted.
\end{proof}

It follows in particular that $R$ is an equivalence relation. From now on, we will write $x \approx y$ for
$x R y$.

\begin{definition}
    \label{MyPreint.neg}
    \lean{MyPreint.neg}
    \leanok
    \uses{MyPreint.R}
We define an operation, called \emph{negation} on $\MyPreint$ as follows: the negation of $x = (a,b)$ is
$(b,a)$:
\[
-x = -(a,b) = (b,a)
\]
\end{definition}

\begin{lemma}
    \label{MyPreint.neg_quotient}
    \lean{MyPreint.neg_quotient}
    \leanok
If $x \approx x'$, then $-x \approx -x'$.
\end{lemma}
\begin{proof}
    \uses{MyPreint.neg}
Let $x = (a,b)$ and $x' = (a',b')$, so by assumption $a + b' = b + a'$. By definition we have
\[
-x=-(a,b)=(b,a) \mbox{ and } -x'=-(a',b')=(b',a')
\]
We need to show $b+a'=b'+a$, which follows immediately from $a + b' = b + a'$.
\end{proof}

\begin{definition}
    \label{MyPreint.add}
    \lean{MyPreint.add}
    \leanok
    \uses{MyPreint.R}
We define an operation, called \emph{addition} on $\MyPreint$ as follows: the addition of $x = (a,b)$
and $y = (b, c)$ is
\[
x + y = (a,b) + (c,d) = (a+c, b + d)
\]
\end{definition}

\begin{lemma}
    \label{MyPreint.add_quotient}
    \lean{MyPreint.add_quotient}
    \leanok
If $x \approx x'$ and $y \approx y'$, then $x + y \approx x' + y'$.
\end{lemma}
\begin{proof}
    \uses{MyPreint.add}
Let $x = (a,b)$, $y = (c,d)$, $x' = (a',b')$ and $y' = (c',d')$ such that $x \approx x'$ and $y \approx y'$. by assumption we have
\[
a + b' = b + a' \mbox{ and } c + d' = d + c'
\]
Adding these two equalities we get
\[
a+b'+c+d'=b+a'+d+c'
\]
and hence
\[
a+c+b'+d'=b+d+a'+c'
\]
that is $x + y = (a+c,b+d) \approx (a'+c',b'+d') = x'+y'$.
\end{proof}

\begin{definition}
    \label{MyPreint.mul}
    \lean{MyPreint.mul}
    \leanok
    \uses{MyPreint.R}
We define an operation, called \emph{multiplication} on $\MyPreint$ as follows: the multiplication of $x = (a,b)$ and $y = (b, c)$ is
\[
x * y = (a,b) * (c,d) = (a*c+b*d, a*d+b*c)
\]
\end{definition}

\begin{lemma}
    \label{MyPreint.mul_quotient}
    \lean{MyPreint.mul_quotient}
    \leanok
If $x \approx x'$ and $y \approx y'$, then $x * y \approx x' * y'$.
\end{lemma}
\begin{proof}
\uses{MyPreint.mul}
Let $x = (a,b)$, $y = (c,d)$, $x' = (a',b')$ and $y' = (c',d')$ such that $x \approx x'$ and $y \approx y'$. by assumption we have
\[
a + b' = b + a' \mbox{ and } c + d' = d + c'
\]
Multiplying the first equality by $c'$ and by $d'$ we get
\begin{equation} \label{H1}
a * c' + b' * c' = b * c' + a' * c'
\end{equation}
and
\begin{equation} \label{H2}
b * d' + a' * d' = a * d' + b' * d'
\end{equation}
Multiplying the second equality by $a$ and by $b$ we get
\begin{equation} \label{H3}
a * c + a * d' = a * d + a * c'
\end{equation}
and
\begin{equation} \label{H4}
b * d + b * c' = b * c + b * d'
\end{equation}
Adding \eqref{H1} and \eqref{H4} we get
\[
a * c' + b' * c' + b * d + b * c' = b * c' + a' * c' + b * c + b * d'
\]
Adding \eqref{H3} and \eqref{H2} we get
\[
 a * c + a * d' + b * d' + a' * d' = a * d + a * c' + a * d' + b' * d'
\]
Adding the last two equations and cancelling the terms appearing on both sides we finally have
\[
b' * c' + b * d + a * c + a' * d' =  a' * c' + b * c + a * d + b' * d'
\]
that is $x * y = \approx  x'*y'$.
\end{proof}